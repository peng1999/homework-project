% Created 2019-01-02 三 17:51
% Intended LaTeX compiler: pdflatex
\documentclass[UTF8]{ctexart}
\usepackage{svg}
\usepackage{graphicx}
\usepackage{grffile}
\usepackage{longtable}
\usepackage{wrapfig}
\usepackage{rotating}
\usepackage[normalem]{ulem}
\usepackage{amsmath}
\usepackage{textcomp}
\usepackage{amssymb}
\usepackage{capt-of}
\usepackage{hyperref}
\author{彭冠文}
\date{\today}
\title{一个脚本语言解释器}
\hypersetup{
 hidelinks=true,
 pdfauthor={Peng Guanwen},
 pdftitle={一个脚本语言解释器},
 pdfkeywords={},
 pdfsubject={},
 pdfcreator={Emacs 26.1 (Org mode 9.1.9)}, 
 pdflang={English}}
\begin{document}

\maketitle
\tableofcontents

这是NPU王岐老师的程序设计课程的自选大作业题目。

\section{语言描述}
\label{sec:org7ac6f48}

\subsection{表达式}
\label{sec:org295ddb2}
支持赋值,四则运算,绝对值,函数调用等。内建常用的数学函数如 \texttt{sin} \texttt{cos} \texttt{tan} 等。
语法:
\begin{verbatim}
exp: IDENT '=' exp         
   | IDENT '(' exp_list ')'
   | exp '+' exp           
   | exp '-' exp           
   | exp '*' exp           
   | exp '/' exp           
   | '|' exp '|'           
   | '(' exp ')'           
   | '-' exp
   | NUMBER                
   | "inf"
   | IDENT                 
\end{verbatim}
例子:
\begin{verbatim}
1 + sin(2) * ||-3 / 4| * 5|
\end{verbatim}

\texttt{print} 函数可以打印值
\begin{verbatim}
a = 1
b = 2
print("a + b =", a + b)
\end{verbatim}

\subsection{变量}
\label{sec:org9149477}
变量无需声明,直接赋值就行了。
\begin{verbatim}
a = 1
\end{verbatim}

\subsection{自定义函数}
\label{sec:org31eea38}
自定义函数的例子如下。注意等号后面必须跟 \textbf{一个} 表达式。
\begin{verbatim}
def add(a, b) = a + b
def sqr(x) = x^2
\end{verbatim}

\subsection{代码块}
\label{sec:orgbd4e733}
代码块用大括号包围,以回车或分号分隔表达式。代码块的值是最后一个表达式的值。
\begin{verbatim}
{ a = 1; a }
{ a = 1
  b = 2
  a + b
}
\end{verbatim}

\subsection{分支和循环}
\label{sec:orgb6267ee}
分别用 \texttt{if} 和 \texttt{while} 语句。注意 \texttt{if} 语句会返回值。在下面的例子中, \texttt{b} 等于 0。
\begin{verbatim}
a = 1
b = if a < 0 { 1 } else { 0 }
sum = 0
while a < 100 {
    sum = sum + a
    a = a + 1
}
\end{verbatim}

\section{实现架构}
\label{sec:org0cb82d6}

由于是C/C++程序设计课,只能使用C++编写。于是我选择了生成C/C++代码的Flex/Bison。
Flex 是词法分析器,用于生成 Token,交给 Bison 具体处理生成抽象语法树(AST)。
AST 由 C++ 在 \texttt{ast.h} 中定义。有一个 \texttt{ast\_node} 抽象基类。由它派生出了所有语法
结构类。 \texttt{ast\_node} 有一个 \texttt{eval(env\_scope)} 虚函数,用于求值。求值的结果用
\texttt{object} 类表示。

\section{代码行数}
\label{sec:org5252de5}

计空白行:731

不计空白:577

\section{使用到的技术}
\label{sec:org0c661ba}

\begin{itemize}
\item Flex/Bison 词法/语法分析器
\item C++ 11 特性 (包括 \texttt{unqiue\_ptr} ,自动类型推断,Lambda 表达式,移动语义,基于范围的 for 循环)
\item C++ 14 特性 (包括泛型 Lambda 表达式)
\item C++ 17 特性 (包括 \texttt{variant} 类,结构化绑定)
\item 面向对象编程,类继承与虚方法
\item 多文件编译与链接
\item 编译系统 CMake 与 GNU Make
\end{itemize}

下面是源代码的依赖关系图
 \begin{center}
\includesvg[width=.9\linewidth]{graph}
\end{center} 
\end{document}
